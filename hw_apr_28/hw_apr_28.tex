\documentclass[12pt]{article}
\textwidth=17cm \oddsidemargin=-0.9cm \evensidemargin=-0.9cm
\textheight=23.7cm \topmargin=-1.7cm

\usepackage{amssymb, amsmath, amsfonts}
\usepackage{moreverb}
\usepackage{graphicx}
\usepackage{enumerate}
\usepackage{graphics}
\usepackage{color}
\usepackage{array}
\usepackage{float}
\usepackage{hyperref}
\usepackage{textcomp}
\usepackage{alltt}
\usepackage{mathtools}
\usepackage{tikz}
\usetikzlibrary{positioning}
\usetikzlibrary{arrows}
\usepackage{pgfplots}
\usepackage{bigints}
\allowdisplaybreaks

\newcommand{\suchthat}{\, \mid \,}
\renewcommand{\theenumi}{\alph{enumi}}
\newcommand\Wider[2][3em]{%
\makebox[\linewidth][c]{%
  \begin{minipage}{\dimexpr\textwidth+#1\relax}
  \raggedright#2
  \end{minipage}%
  }%
}

\setcounter{section}{-1}

\title{\bf HW: Custom Problem, Section 7A \#1, 2, 7, 13}
\author{\bf Sam Fleischer}
\date{\bf April 28, 2015}

\begin{document}
{\bf MATH 462 \hfill Advanced Linear Algebra \ \ \ \ \ \hfill Spring 2015} 

{\let\newpage\relax\maketitle}

\section*{Custom Problems}
\subsection*{\#1}
{\it Let $T \in \mathcal{L}(V)$.  Prove $T$ is invertible if and only if $T^*$ is invertible.} \\

\noindent ``$\Longrightarrow$'' \\
Suppose $T$ is invertible, and that $T^*(w_1) = T^*(w_2)$.  Then by the definition of $T^*$
\begin{align*}
	\langle T(v), w_1\rangle = \langle v, T^*(w_1)\rangle \ \ \ \text{and}\ \ \ \langle T(v), w_2\rangle = \langle v, T^*(w_2)\rangle
\end{align*}
for every $v \in V$.  Thus,
\begin{align*}
	\langle T(v), w_1 \rangle &= \langle v, T^*(w_1) \rangle = \langle v, T^*(w_2) \rangle = \langle T(v), w_2 \rangle
\end{align*}
for every $v \in V$.  However, since $T$ is invertible, it is surjective, and thus $\text{range}(T) = V$, and so 
\begin{align*}
	\langle z, w_1 \rangle = \langle z, w_2 \rangle
\end{align*}
for every $z \in V$.  Thus $w_1 = w_2$.  Thus $T^*$ is injective.  Thus $T^*$ is invertible. \\

\noindent ``$\Longleftarrow$'' \\
Suppose $T^*$ is invertible.  Then by the above argument, $(T^*)^*$ is invertible.  However, $(T^*)^* = T$.  Thus $T$ is invertible. \\

\noindent Thus $T$ is invertible if and only if $T^*$ is invertible. \hfill $\square$

\section*{Section 7A}
\subsection*{\#1}
{\it Suppose $n$ is a positive integer.  Define $T \in \mathcal{L}(\mathbb{F}^n)$ by}
\begin{align*}
	T(z_1, \dots, z_n) = (0, z_1, \dots, z_{n-1})
\end{align*}
{\it Find a formula for $T^*(w_1, \dots, w_n)$.} \\

\noindent Let $w = (w_1, \dots, w_n) \in \mathcal{L}(\mathbb{F}^n)$.  Note the following:
\begin{align*}
	\langle T(z_1, \dots, z_n), w \rangle &= \langle (0, z_1, \dots, z_{n-1}), (w_1, \dots, w_n) \rangle \\
	&= z_1w_2 + \dots z_{n-1}w_n \\
	&= \langle (z_1, dots, z_n), (w_2, \dots, w_n, 0) \rangle
\end{align*}
for every $v \in V$.  Thus, by the definition of $T^*$,
\begin{align*}
	T^*(w_1, \dots, w_n) = (w_2, \dots, w_n, 0)
\end{align*}
\hfill $\square$

\subsection*{\#2}
{\it Suppose $T \in \mathcal{L}(V)$ and $\lambda \in \mathbb{F}$.  Prove that $\lambda$ is an eigenvalue of $T$ if and only if $\overline{lambda}$ is an eigenvalue of $T^*$.} \\

\noindent ``$\Longrightarrow$'' \\
Suppose $\lambda$ is an eigenvalue of $T$.  Then $T(v) = \lambda v$ for some nonzero $v \in V$.  Thus $v \in \text{null}(T - \lambda I)$.  In other words, $(T - \lambda I)(v) = 0$.  Now note that
\begin{align*}
	(T - \lambda I)^* = T^* - (\lambda I)^* = T^* - \overline{\lambda}I^* = T^* - \overline{\lambda}I
\end{align*}
since $I$ is self-adjoint.  Thus,
\begin{align*}
	\langle (T - \lambda I)(v), w \rangle = \langle v, (T^* - \overline{\lambda}I)(w) \rangle
\end{align*}
for all $w \in V$.  However, $(T - \lambda I)(v) = 0$.  Then
\begin{align*}
	\langle (T - \lambda I)(v), w \rangle \langle 0, w \rangle = 0 \\
	\implies 0 = \langle v, (T^* - \overline{\lambda}I)(w) \rangle
\end{align*}
for all $w \in V$.  Now suppose $\overline{\lambda}$ is {\it not} an eigenvalue of $T^*$.  Then $\text{range}(T^* - \overline{\lambda}I) = V$.  Then in particular, $\exists z \in V$ such that $(T^* - \overline{\lambda}I)(z) = v$.  Thus
\begin{align*}
	0 = \langle v, v \rangle
\end{align*}
Thus $v = 0$, which implies $v$ is not an eigenvalue of $T$, which is a contradiction.  Then $\overline{\lambda}$ is an eigenvalue of $T^*$. \\

\noindent ``$\Longleftarrow$''
Suppose $\overline{\lambda}$ is an eigenvalue of $T^*$.  Then since $\overline{\overline{\lambda}} = \lambda$ and $(T^*)^* = T$, then by the above argument, $\lambda$ is an eigenvalue of $T$. \\

\noindent Thus $\lambda$ is an eigenvalue of $T$ if and only if $\overline{\lambda}$ is an eigenvalue of $T^*$. \hfill $\square$

\subsection*{\#7}
{\it Suppose $S, T \in \mathcal{L}(V)$ are self-adjoint.  Prove that $ST$is self-adjoint if and only if $ST = TS$.} \\

\noindent Since $S$ and $T$ are self-adjoint, $S = S^*$ and $T = T^*$.  Note the following:
\begin{align*}
	(ST) = (ST)^* \iff ST = (T^*)(S^*) \iff ST = (T)(S)	\iff ST = TS
\end{align*}
Thus, $ST$ is self-adjoint if and only if $ST = TS$. \hfill $\square$

\subsection*{\#13}
{\it Give an example of an operator $T \in \mathcal{\mathbb{C}^4}$ such that $T$ is normal but not self-adjoint.} \\

\noindent In other words, we want to find an operator such that $TT^* = T^*T$, but $T \neq T^*$.  Let $e = (e_1, e_2, e_3, e_4)$ be the standard basis in $\mathbb{C}^4$.  Then define $T(z_1, z_2, z_3, z_4) = (z_1, z_2, iz_3, -iz_4)$.  In other words,
\begin{align*}
	\mathcal{M}(T, e) = \left(\begin{array}{cccc}
		1 & 0 & 0 & 0 \\
		0 & 1 & 0 & 0 \\
		0 & 0 & i & 0 \\
		0 & 0 & 0 & -i
	\end{array}\right)
\end{align*}
Since $\mathcal{M}(T^*, e)$ is the conjugate transpose of $\mathcal{M}(T, e)$,
\begin{align*}
	\mathcal{M}(T^*, e) = \left(\begin{array}{cccc}
		1 & 0 & 0 & 0 \\
		0 & 1 & 0 & 0 \\
		0 & 0 & -i & 0 \\
		0 & 0 & 0 & i
	\end{array}\right)
\end{align*}
Note
\begin{align*}
	\mathcal{M}(T, e) \cdot \mathcal{M}(T^*, e) &= \left(\begin{array}{cccc}
		1 & 0 & 0 & 0 \\
		0 & 1 & 0 & 0 \\
		0 & 0 & i & 0 \\
		0 & 0 & 0 & -i
	\end{array}\right) \cdot \left(\begin{array}{cccc}
		1 & 0 & 0 & 0 \\
		0 & 1 & 0 & 0 \\
		0 & 0 & -i & 0 \\
		0 & 0 & 0 & i
	\end{array}\right) \\[.1cm]
	&= \left(\begin{array}{cccc}
		1 & 0 & 0 & 0 \\
		0 & 1 & 0 & 0 \\
		0 & 0 & -1 & 0 \\
		0 & 0 & 0 & -1
	\end{array}\right) \\[.1cm]
	&= \left(\begin{array}{cccc}
		1 & 0 & 0 & 0 \\
		0 & 1 & 0 & 0 \\
		0 & 0 & -i & 0 \\
		0 & 0 & 0 & i
	\end{array}\right) \cdot \left(\begin{array}{cccc}
		1 & 0 & 0 & 0 \\
		0 & 1 & 0 & 0 \\
		0 & 0 & i & 0 \\
		0 & 0 & 0 & -i
	\end{array}\right) \\[.1cm]
	&= \mathcal{M}(T^*, e) \cdot \mathcal{M}(T, e) \\
	\implies TT^* &= T^*T
\end{align*}
Thus $T$ is normal.  However, $\mathcal{M}(T, e) \neq \mathcal{M}(T^*, e)$, and so $T \neq T^*$.  Thus $T$ is not self-adjoint.

\end{document}
