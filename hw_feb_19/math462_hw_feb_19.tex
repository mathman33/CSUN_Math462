\documentclass[12pt]{article}

\usepackage{amssymb, amsmath, amsfonts}
\usepackage{amsthm}
\usepackage{moreverb}
\usepackage{graphicx}
\usepackage{enumerate}
\usepackage{graphics}
\usepackage{color}
\usepackage{array}
\usepackage{float}
\usepackage{hyperref}
\usepackage{textcomp}
\usepackage{alltt}
\usepackage{mathtools}
\usepackage[T1]{fontenc}
\usepackage{fullpage}
\usepackage{changepage}
\usepackage{tikz}
\usepackage[utf8]{inputenc}
\newcommand{\suchthat}{\, \mid \,}
\allowdisplaybreaks
\def\arraystretch{1.3}

\begin{document}

{\bf MATH 462 \hfill Advnaced Linear Algebra \hfill Spring 2015}

\title{\bf Homework: Sec. 3B \# 17, Sec. 3C \# 2}
\author{\bf Sam Fleischer}
\date{\bf Due: Thurs. Feb. 19, 2015}

{\let\newpage\relax\maketitle}
\maketitle

\section*{Sec. 3B}
\subsubsection*{\#17}
{\it Suppose $V$ and $W$ are both finite-dimensional.  Prove that there exists an injective linear map from $V$ to $W$ if and only if $\text{dim}(V) \leq \text{dim}(W)$.} \\

\noindent Suppose $T \in \mathcal{L}(V, W)$ such that $T$ is injective.  Since $T(0) = 0$ and $T$ is injective, $\text{null}(T) = \{0\}$.  Thus $\text{dim}(\text{null}(T)) = 0$.  Thus,
\begin{align*}
\text{dim}(\text{range}(V)) &= \text{dim}(V) - \text{dim}(\text{null}(T)) \\
\implies \text{dim}(\text{range}(V)) &= \text{dim}(V) - 0 \\
\implies \text{dim}(\text{range}(V)) &= \text{dim}(V)
\end{align*}
Since $\text{range}(V) \subset W$, $\text{dim}(\text{range}(V)) \leq \text{dim}(W)$.  Thus $\text{dim}(V) \leq \text{dim}(W)$. \\

\noindent Now suppose $m = \text{dim}(V) \leq \text{dim}(W) = n$.  Then let $(v_1, \dots, v_m)$ be a basis for $V$ and let $(w_1, \dots, w_n)$  be a basis for $W$.  Define $T:V\rightarrow W$ by $T(v_i) = w_i$ for $i \in \{1, \dots, m\}$.  Note this is a valid map since $m \leq n$ (so $w_i \in W$ for $i \in \{1, \dots, m\}$).  Also, this is a linear map because every linear map is uniquely defined by where it sends basis vectors.  Thus $T \in \mathcal{L}(V, W)$.  Now suppose $a, b \in V$ and $T(a) = T(b)$.  Then $\exists!a_1, \dots, a_m, b_1, \dots, b_m \in \mathbb{F}$ such that $a = a_1v_1 + \dots + a_mv_m$ and $b = b_1v_1 + \dots + b_mv_m$.  Then by the additivity and homogeneity of $T$, $T(a) = T(a_1v_1 + \dots + a_mv_m) = a_1w_1 + \dots + a_mw_m$ and $T(b) = T(b_1v_1 + \dots + b_mv_m) = b_1w_1 + \dots + b_mw_m$.  However, since $T(a) = T(b)$, $a_1w_1 + \dots + a_mw_m = b_1w_1 + \dots + b_mw_m \implies (a_1 - b_1)w_1 + \dots (a_m - b_m)w_m$ = 0.  Since $(w_1, \dots, w_n)$ is a basis, $(w_1, \dots, w_m)$ is linearly independent.  And thus $(a_i - b_i) = 0$ and $a_i = b_i$ for $i \in \{1, \dots, m\}$.  Thus $a = b$.  Thus $T$ is injective. \\

\noindent Thus for finite dimensional vector spaces $V$ and $W$, $\exists T \in \mathcal{L}(V, W)$ such that $T$ is injective $\iff \text{dim}(V) \leq \text{dim}(W)$. \qedsymbol

\section*{Sec. 3C}
\subsubsection*{\#2}
{\it Suppose $D \in \mathcal{L}(\mathcal{P}_3(\mathbb{R}), \mathcal{P}_2(\mathbb{R}))$ is the differentiation map defined by $Dp = p'$.  Find a basis of $\mathcal{P}_3(\mathbb{R})$ and a basis of $\mathcal{P}_2(\mathbb{R})$ such that the matrix of $D$ with respect to these bases is}
\begin{align*}
\left(\begin{array}{cccc}
1 & 0 & 0 & 0 \\
0 & 1 & 0 & 0 \\
0 & 0 & 1 & 0
\end{array}\right)
\end{align*}

\noindent Consider the list of vectors in $\mathcal{P}_3(\mathbb{R})$, $\mathcal{X} = \left(x^3, x^2, x, 1\right)$, and let $p = p_3x^3 + p_2x^2 + p_1x + p_0 \in \mathcal{P}_3(\mathbb{R})$.  Then let $a_i = p_i$ for $i \in \{0, 1, 2, 3\}$.  Then $a_3(x^3) + a_2(x^2) + a_1(x) + a_0(1) = p$, and thus $\mathcal{X}$ spans $\mathcal{P}_3(\mathbb{R})$.  Then suppose $a_3x^3 + a_2x^2 + a_1x + a_0 = 0 = 0x^3 + 0x^2 + 0x + 0$.  This implies $a_3 = a_2 = a_1 = a_0 = 0$, and thus $\mathcal{X}$ is linearly independent and a basis of $\mathcal{P}_3(\mathbb{R})$.  Consider the list of vectors in $\mathcal{P}_2(\mathbb{R})$, $\mathcal{Y} = (3x^2, 2x, 1)$, and let $q = q_2x^2 + q_1x + q_0 \in \mathcal{P}_2(\mathbb{R})$.  Then let $b_2 = \frac{1}{3}q_2$, $b_1 = \frac{1}{2}q_1$, and $b_0 = q_0$.  Then $b_2(3x^2) + b_1(2x) + b_0(1) = q$, and thus $\mathcal{Y}$ spans $\mathcal{P}_2(\mathbb{R})$.  Then suppose $b_2(3x^2) + b_1(2x) + b_0(1) = 0 = 0x^2 + 0x + 0$.  This implies $3b_2 = 2b_1 = b_0 = 0$, which implies $b_2 = b_1 = b_0 = 0$.  Thus $\mathcal{Y}$ is linearly independent and a basis of $\mathcal{P}_2(\mathbb{R})$.  Then,
\begin{align*}
D(x_3) &= 3x^2 = 1(3x^2) + 0(2x) + 0(1)\\
D(x_2) &= 2x = 0(3x^2) + 1(2x) + 0(1)\\
D(x) &= 1 = 0(3x^2) + 0(2x) + 1(1)\\
D(1) &= 0 = 0(3x^2) + 0(2x) + 0(1)
\end{align*}
Thus,
\begin{align*}
\mathcal{M}(D, \mathcal{X}, \mathcal{Y}) = \left(\begin{array}{cccc}
1 & 0 & 0 & 0 \\
0 & 1 & 0 & 0 \\
0 & 0 & 1 & 0
\end{array}\right)
\end{align*}


%\pagebreak
%\begin{thebibliography}{99}
%
%\bibitem{Abrams1997b}
%Abrams, P.~A. and Matsuda, H.
%Prey Adaptation as a Cause of Predator-Prey Cycles.
%\emph{Evolution}
%1997, 51:1742-1750.
%
%\bibitem{Chavez2001}
%Brauer, F., Castillo-Chavez, C.
%Mathematical Models in Population Biology and Epidemiology.
%Springer,
%2011. Print.
%
%\bibitem{Boyce2012}
%Boyce, W. E., and DiPrima, R. C.
%Elementary Differential Equations and Boundary Value Problems %10\textsuperscript{th} ed.
%Wiley Global Education
%2012. Print.
%
%\bibitem{Saloniemi1993}
%Saloniemi, I.
%A Coevolutionary Predator-Prey Model with Quantitative Characters.
%\emph{American Naturalist}
%1993, 141:880-896.
%
%\bibitem{Schreiber2011}
%Schreiber, S.~J., B$\ddot{\mbox{u}}$rger,  R., and Bolnick,  D.~I.
%The Community Effects of Phenotypic and Genetic Variation within a Predator %Population.
%\emph{Ecology}
%2011,  92(8):526-543. 
%
%\end{thebibliography}

\end{document}