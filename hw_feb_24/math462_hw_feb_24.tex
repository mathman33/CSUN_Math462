\documentclass[12pt]{article}

\usepackage{amssymb, amsmath, amsfonts}
\usepackage{amsthm}
\usepackage{moreverb}
\usepackage{graphicx}
\usepackage{enumerate}
\usepackage{graphics}
\usepackage{color}
\usepackage{array}
\usepackage{float}
\usepackage{hyperref}
\usepackage{textcomp}
\usepackage{alltt}
\usepackage{mathtools}
\usepackage[T1]{fontenc}
\usepackage{fullpage}
\usepackage{changepage}
\usepackage{tikz}
\usepackage[utf8]{inputenc}
\newcommand{\suchthat}{\, \mid \,}
\allowdisplaybreaks
\def\arraystretch{1.3}

\begin{document}

{\bf MATH 462 \hfill Advnaced Linear Algebra \hfill Spring 2015}

\title{\bf Homework: Custom Problem, Sec. 3C \# 4}
\author{\bf Sam Fleischer}
\date{\bf Due: Tues. Feb. 24, 2015}

{\let\newpage\relax\maketitle}
\maketitle

\section*{Custom Problem}
{\it Let $U = \{(x, y, z) \in \mathbb{F}^3 \suchthat x + y + z = 0\}$, $f_1 = (1, -1, 0)$, and $f_2 = (1, 0, -1)$.  Let $T \in \mathcal{L}(U, U)$ defined by $T(f_1) = (1, 1, -2)$, and $T(f_2) = (1, 0, -1)$.}
\begin{enumerate}[\ \ \ (a)\ \ ]
\item {\it Compute $\mathcal{M}(T, (f_1, f_2), (f_1, f_2))$, $\mathcal{M}((-4, 2, 2), (f_1, f_2))$, and $\mathcal{M}(T(-4, 2, 2), (f_1, f_2))$.}\\

\noindent Note $T(f_1) = (1, 1, -2) = -1(1, -1, 0)+2(1, 0, -1) = (-1)f_1 + (2)f_2$, and $T(f_2) = (1, 0 ,-1) = 0(1, -1, 0) + 1(1, 0, -1) = (0)f_1 + (1)f_2$.  Thus,
\begin{align*}
\mathcal{M}(T, (f_1, f_2), (f_1, f_2)) = \left(\begin{array}{cc}
-1 & 0 \\
2 & 1
\end{array}\right)
\end{align*}
Note $(-4, 2, 2) = -2(1, -1, 0) -2(1, 0, -1) = (-2)f_1 + (-2)f_2$.  Thus,
\begin{align*}
\mathcal{M}((-4, 2, 2), (f_1, f_2)) = \left(\begin{array}{c}
-2\\
-2
\end{array}\right)
\end{align*}
Note $T(-4, 2, 2) = T((-2)f_1 + (-2)f_2) = (-2)T(f_1) + (-2)T(f_2) = (-2)(1, 1, -2) + (-2)(1, 0, -1) = (-4, -2, 6) = 2(1, -1, 0) + (-6)(1, 0, -1) = (2)f_1 + (-6)f_2$.  Thus,
\begin{align*}
\mathcal{M}(T(-4, 2, 2), (f_1, f_2)) = \left(\begin{array}{c}
2\\
-6
\end{array}\right)
\end{align*}

\item {\it Comfirm $\mathcal{M}(T(-4, 2, 2), (f_1, f_2)) = \mathcal{M}(T, (f_1, f_2), (f_1, f_2)) \cdot \mathcal{M}((-4, 2, 2), (f_1, f_2))$}.
\begin{align*}
\left(\begin{array}{cc}
-1 & 0 \\
2 & 1
\end{array}\right)
\cdot
\left(\begin{array}{c}
-2\\
-2
\end{array}\right)
=
\left(\begin{array}{c}
(-1)(-2) + (0)(-2) \\
(2)(-2) + (1)(-2)
\end{array}\right)
=
\left(\begin{array}{c}
2 \\
-6
\end{array}\right)
\end{align*}
This helps confirm that if $T \in \mathcal{L}(V, W)$, and if $\mathcal{V}$ and $\mathcal{W}$ are bases for $V$ and $W$, respectively, and if $v \in V$, that
\begin{align*}
\mathcal{M}(T(v), \mathcal{W}) = \mathcal{M}(T, \mathcal{V}, \mathcal{W}) \cdot \mathcal{M}(v, \mathcal{V})
\end{align*}

\end{enumerate}

\section*{Sec. 3C}
\subsection*{\# 4}
{\it Suppose $\mathcal{V} = (v_1, \dots, v_m)$ is a basis of $V$ and $W$ is finite-dimensional.  Suppose $T \in \mathcal{L}(V, W)$.  Prove that there exists a basis $(w_1, \dots w_n)$ of $W$ such that all the entries in the first column of $\mathcal{M}(T)$ (with respect to the bases $(v_1, \dots, v_m)$ and $(w_1, \dots, w_n)$) are $0$ except for possibly a $1$ in the first row, first column.} \\

\noindent
Since $W$ is finite dimensional, let $\text{dim}(W) = n$, and let $\mathcal{U} = (u_1, \dots, u_n)$ be a basis for $W$.  Then since $T \in \mathcal{L}(V, W)$, $T(v_1) = b_1u_1 + \dots + b_nu_n$ for some $b_1, \dots, b_n \in \mathbb{F}$.
\subsubsection*{Case 1: $b_1 = \dots = b_n = 0$}
Then $T(v_1) = 0u_1 + \dots + 0u_n$.  Thus if $\mathcal{M}(T, \mathcal{V}, \mathcal{U}) = [a_{i,j}]_{n\times m}$, then $a_{i,1} = 0$ for $i = 1, \dots, n$.  In other words, all the entries in the first column of $\mathcal{M}(T, \mathcal{V}, \mathcal{U})$ are $0$.
\subsubsection*{Case 2: $\exists k \in \{1, \dots, n\}$ {\rm such that} $b_k \neq 0$}
Then construct the list $\mathcal{W} = (w_1, \dots, w_n)$ where $w_k = T(v_1) = b_1u_1 + \dots + b_nu_n$ and $w_i = u_i$ for $i \in \{1, \dots, k-1, k+1, \dots, n\}$.  Then let
\begin{align*}
a_1w_1 + \dots + a_nw_n = 0
\end{align*}
for some $a_1, \dots, a_n \in \mathbb{F}$.  Thus
\begin{align*}
a_1u_1 + \dots a_{k-1}u_{k-1} + a_k(b_1u_1 + \dots + b_nu_n) + a_{k+1}u_{k+1} + \dots + a_nu_n &= 0
\end{align*}
or
\begin{align*}
(a_1 + a_kb_1)u_1 + \dots + (a_{k-1} + a_kb_{k-1})u_{k-1} + a_kb_ku_k + (a_{k+1} + a_kb_{k+1})u_{k+1}&\\
 + \dots + (a_n + a_kb_n)u_n &= 0
\end{align*}
But since $\mathcal{U}$ is a basis (and therefore linearly independent), all of the coefficients must be equivalently $0$.  In particular, $a_kb_k = 0$.  By assumption, $b_k \neq 0$.  Thus $a_k = 0$ since $\mathbb{F}$ is a field and there are no zero-divisors in fields.  $a_k = 0 \implies a_1 = \dots = a_n = 0$.  Thus $\mathcal{W}$ is linearly independent.  However, since $\text{dim}(W) = n$, all linearly independent lists in $W$ of length $n$ form a basis of $W$.  Thus $\mathcal{W}$ is a basis of $W$.  Now re-order $\mathcal{W}$ in the following way: $\mathcal{Z} = (w_k, w_1, \dots, w_{k-1}, w_{k+1}, \dots, w_n)$.  Note $\mathcal{Z}$ is a basis because it is simply a re-ordering of another basis.  Then $T(v_1) = w_k = 1w_k + 0w_1 + \dots + 0w_{k-1} + 0w_{k+1} + \dots + 0w_n$.  Thus if $\mathcal{M}(T, \mathcal{V}, \mathcal{Z}) = [a_{i,j}]_{n\times m}$, then $a_{1,1} = 1$ and $a_{i,1} = 0$ for $i = 2, \dots, n$.  In other words, all the entries in the first column of $\mathcal{M}(T, \mathcal{V}, \mathcal{Z})$ are $0$ except for a $1$ in the first row, first column. \\

\noindent In either case, there exists a basis $(w_1, \dots w_n)$ of $W$ such that all the entries in the first column of $\mathcal{M}(T)$ (with respect to the bases $(v_1, \dots, v_m)$ and $(w_1, \dots, w_n)$) are $0$ except for possibly a $1$ in the first row, first column. \qedsymbol

%\pagebreak
%\begin{thebibliography}{99}
%
%\bibitem{Abrams1997b}
%Abrams, P.~A. and Matsuda, H.
%Prey Adaptation as a Cause of Predator-Prey Cycles.
%\emph{Evolution}
%1997, 51:1742-1750.
%
%\bibitem{Chavez2001}
%Brauer, F., Castillo-Chavez, C.
%Mathematical Models in Population Biology and Epidemiology.
%Springer,
%2011. Print.
%
%\bibitem{Boyce2012}
%Boyce, W. E., and DiPrima, R. C.
%Elementary Differential Equations and Boundary Value Problems %10\textsuperscript{th} ed.
%Wiley Global Education
%2012. Print.
%
%\bibitem{Saloniemi1993}
%Saloniemi, I.
%A Coevolutionary Predator-Prey Model with Quantitative Characters.
%\emph{American Naturalist}
%1993, 141:880-896.
%
%\bibitem{Schreiber2011}
%Schreiber, S.~J., B$\ddot{\mbox{u}}$rger,  R., and Bolnick,  D.~I.
%The Community Effects of Phenotypic and Genetic Variation within a Predator %Population.
%\emph{Ecology}
%2011,  92(8):526-543. 
%
%\end{thebibliography}

\end{document}