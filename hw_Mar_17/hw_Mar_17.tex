\documentclass[12pt]{article}
\textwidth=17cm \oddsidemargin=-0.9cm \evensidemargin=-0.9cm
\textheight=23.7cm \topmargin=-1.7cm

\usepackage{amssymb, amsmath, amsfonts}
\usepackage{moreverb}
\usepackage{graphicx}
\usepackage{enumerate}
\usepackage{graphics}
\usepackage{color}
\usepackage{array}
\usepackage{float}
\usepackage{hyperref}
\usepackage{textcomp}
\usepackage{alltt}
\usepackage{mathtools}
\usepackage{tikz}
\usetikzlibrary{positioning}
\usetikzlibrary{arrows}
\usepackage{pgfplots}
\usepackage{bigints}

\newcommand{\suchthat}{\, \mid \,}
\renewcommand{\theenumi}{\alph{enumi}}
\newcommand\Wider[2][3em]{%
\makebox[\linewidth][c]{%
  \begin{minipage}{\dimexpr\textwidth+#1\relax}
  \raggedright#2
  \end{minipage}%
  }%
}

\setcounter{section}{-1}

\title{\bf HW: Custom Problem, Section 5A \#19, 21}
\author{\bf Sam Fleischer}
\date{\bf March 17, 2015}

\begin{document}
{\bf MATH 462 \hfill Advanced Linear Algebra \ \ \ \ \ \hfill Spring 2015} 

{\let\newpage\relax\maketitle}

\section*{Custom Problem}
{\it Let $T \in \mathcal{L}(\mathbb{C}^2)$ by $T(z, w) = (2w, -8z)$.}
	\begin{enumerate}[\it\ \ (a)\ \ ]
		\item {\it Find a basis for $\mathbb{C}^2$ consisting of eigenvectors of $T$.}\\
		
		\noindent To find the eigenvalues, set $T(z, w) = \lambda(z, w)$.
		\begin{align*}
			T(z, w) &= \lambda(z, w) \\
			\implies (2w, -8z) &= (\lambda z, \lambda w) \\
			\implies 2w = \lambda z\text{,} \ \ \ \ &\text{and}\ \ \ \ -8z = \lambda w \\
			\implies -8z &= \frac{\lambda^2}{2}z
		\end{align*}
		If $z = 0$, then $w = 0$.  Then $\lambda$ would not be an eigenvalue.  So we suppose $z \neq 0$, and thus
		\begin{align*}
			-8 &= \frac{\lambda^2}{2} \\
			\implies -16 &= \lambda^2 \\
			\implies \lambda &= \pm 4i
		\end{align*}
		Thus the two eigenvalues for $T$ are $4i$ and $-4i$.  To find their eigenspaces, set $T(z, w) = 4i(z, w)$ and $T(z, w) = -4i(z, w)$, respectively.  So,
		\begin{align*}
			T(z, w) &= 4i(z, w) \\
			\implies (2w, -8z) &= (4iz, 4iw) \\
			\implies w &= 2iz
		\end{align*}
		Thus $(1, 2i)$ is an eigenvector corresponding to $4i$.  Furthermore, $(1, 2i)$ is a basis for $E(T, 4i)$.  Also,
		\begin{align*}
			T(z, w) &= -4i(z, w) \\
			\implies (2w, -8z) &= (-4iz, -4iw) \\
			\implies w &= -2iz
		\end{align*}
		Thus $(1, -2i)$ is an eigenvector corresponding to $-4i$.  Furthermore, $(1, -2i)$ is a basis for $E(T, -4i)$.  Then $\pi = ((1, 2i), (1, -2i))$ is a linearly independent set since eigenvectors corresponding to different eigenvalues are linearly independent.  Since $\text{dim}(\mathbb{C}^2) = 2$ and $\text{len}(\pi) = 2$, $\pi$ is a basis for $\mathbb{C}^2$ consisting of eigenvectors of $T$.
		\item {\it Find $\mathcal{M}(T)$ with respect to this basis.}\\
		
		\noindent Since
		\begin{align*}
			T(1, 2i) &= (2(2i), -8(1)) \\
			&= (4i, -8) \\
			&= 4i(1, 2i) + 0(1, -2i)
		\end{align*}
		and
		\begin{align*}
		T(1, -2i) &= (2(-2i), -8(1)) \\
		&= (-4i, -8) \\
		&= 0(1, 2i) -4i(1, -2i)
		\end{align*}
		then 
		\begin{align*}
			\mathcal{M}(T, \pi) = \left(\begin{array}{cc}
				4i & 0 \\
				0 & -4i
			\end{array}\right)
		\end{align*}
	\end{enumerate}
	
\section*{5A}
\subsection*{\#19}
	{\it Suppose $n$ is a positive integer and $T \in \mathcal{L}(\mathbb{F}^n)$ is defined by}
	\begin{align*}
		T(x_1, \dots, x_n) = (x_1 + \dots x_n, \dots, x_1 + \dots, x_n)
	\end{align*}
	{\it in other words, $T$ is the operator whose matrix (with respect to the standard basis) consists of all $1$'s.  Find the eigenvalues and eigenvectors of $T$.}\\
	
	\noindent To find the eigenvalues, set $T(x_1, \dots, x_n) = \lambda(x_1, \dots, x_n)$.
	\begin{align*}
		T(x_1, \dots, x_n) &= \lambda(x_1, \dots, x_n) \\
		\implies (x_1 + \dots + x_n, \dots, x_1 + \dots + x_n) &= (\lambda x_1, \dots, \lambda x_n) \\
		\implies x_1 + \dots + x_n &= \lambda x_1 = \dots = \lambda x_n\\
		\implies \lambda &= \frac{x_1 + \dots + x_n}{x_1} = \dots = \frac{x_1 + \dots + x_n}{x_n}
	\end{align*}
	If $\lambda \neq 0$, then $x_1 = \dots = x_n$, and thus $(1, \dots, 1)$ is a basis for $E(T, \lambda)$.  However, if $\lambda = 0$, then $x_1 + \dots + x_n = 0$.  Then $E(T, 0) = \{(x_1, \dots, x_n) \in \mathbb{F}^n \suchthat x_1 + \dots + x_n = 0\}$.  Note $\text{dim}(E(T, 0)) = n-1$ and a basis for $E(T, 0)$ is
	\begin{align*}
		\pi = ((1, -1, 0, \dots, 0), (1, 0, -1, 0, \dots, 0), \dots, (1, 0, \dots, 0, -1))
	\end{align*}
\subsection*{\#21}
{\it Suppose $T \in \mathcal{L}(V)$ is invertible.}
	\begin{enumerate}[\it\ \ (a)\ \ ]
		\item {\it Suppose $\lambda \in \mathbb{F}$ with $\lambda \neq 0$.  Prove that $\lambda$ is an eigenvalue of $T$ if and only if $\dfrac{1}{\lambda}$ is an eigenvalue of $T^{-1}$.} \\
		\begin{align*}
			&\lambda \text{ is an eigenvalue of }T \\[.1cm]
			\iff &\exists v \in V \text{such that }T(v) = \lambda v \\[.1cm]
			\iff &T^{-1}(T(v)) = T^{-1}(\lambda v) \\[.1cm]
			\iff &v = \lambda T^{-1}(v) \\[.1cm]
			\iff &T^{-1}(v) = \frac{1}{\lambda}v \text{ for some $v \in V$}\\[.1cm]
			\iff &\frac{1}{\lambda} \text{ is an eigenvalue of }T^{-1}
		\end{align*}\hfill $\square$
		\item {\it Prove that $T$ and $T^{-1}$ have the same eigenvectors.} \\
		
		\noindent Let $\hat{v}$ be an eigenvector of $T$ corresponding to an arbitrary eigenvalue of $T$'s, say $\hat{\lambda}$.  Then
		\begin{align*}
			T(\hat{v}) &= \hat{\lambda}\hat{v} \\
			\implies T^{-1}(T(\hat{v})) &= T^{-1}(\hat{\lambda}\hat{v}) \\
			\implies \hat{v} &= \hat{\lambda}T^{-1}(\hat{v}) \\
			\implies T^{-1}(\hat{v}) &= \frac{1}{\hat{\lambda}}\hat{v} \\
		\end{align*}
		Thus $\hat{v}$ is an eigenvector of $T^{-1}$ corresponding to $\dfrac{1}{\hat{\lambda}}$.  Thus any eigenvector of $T$ is an eigenvector of $T^{-1}$.  However, the same argument and $(T^{-1})^{-1} = T$ implies any eigenvector of $T^{-1}$ is an eigenvector of $T$.  Thus $T$ and $T^{-1}$ have the same eigenvectors. \hfill $\square$
	\end{enumerate}

\end{document}
