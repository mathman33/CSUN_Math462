\documentclass[12pt]{article}

\usepackage{amssymb, amsmath, amsfonts}
\usepackage{amsthm}
\usepackage{moreverb}
\usepackage{graphicx}
\usepackage{enumerate}
\usepackage{graphics}
\usepackage{color}
\usepackage{array}
\usepackage{float}
\usepackage{hyperref}
\usepackage{textcomp}
\usepackage{alltt}
\usepackage{mathtools}
\usepackage[T1]{fontenc}
\usepackage{fullpage}
\usepackage{changepage}
\usepackage{tikz}
\usepackage[utf8]{inputenc}
\newcommand{\suchthat}{\, \mid \,}
\allowdisplaybreaks
\def\arraystretch{1.3}

\begin{document}

{\bf MATH 462 \hfill Advnaced Linear Algebra \hfill Spring 2015}

\title{\bf Homework: Sec. 3A \# 10, Sec. 3B \# 3, 5, 6}
\author{\bf Sam Fleischer}
\date{\bf Due: Tues. Feb. 17, 2015}

{\let\newpage\relax\maketitle}
\maketitle

\section*{Sec. 3A}
\subsubsection*{\#10}
{\it Suppose $U$ is a subspace of $V$ with $U \neq V$.  Suppose $S \in \mathcal{L}(U, W)$ and $S \neq 0$.  Define $T: V\rightarrow W$ by}
\begin{align*}
T(v) = \left\{\begin{array}{ll}S(v) & \text{\ \ if } v \in U \\
0 & \text{\ \ if } v \in V \setminus U
\end{array}\right.
\end{align*}
{\it Prove that $T$ is not a linear map on $V$.} \\

\noindent (The notation $\oplus_V$ and $\oplus_W$ represent the addition operations in $V$ and $W$, respectively.  Since $U$ is a subspace of $V$, the addition operation on $U$ is inherited from $V$.)  Let $u \in U$ such that $S(u) = w \neq 0$, and let $v \in V \setminus U$.  Then since $U \subset V$, $u \in V$.  Then $u \oplus_V v \in V$ since $V$ is closed under $\oplus_V$.  Note $u \oplus_V v \notin U$.
\begin{adjustwidth}{2cm}{}
{\it Suppose $u \oplus_V v \in U$.  Then $u \oplus_V v = \overline{u}$ for some $\overline{u} \in U$.  Then $v = (\text{-}u) \oplus_V \overline{u} \implies v \in U \Longrightarrow\Longleftarrow$ since $v \in V \setminus U$.}
\end{adjustwidth}
Thus $u \oplus_V v \in V \setminus U$.  Then $T(u \oplus_V v) = 0$.  But $T(u) \oplus_W T(v) = S(u) \oplus_W 0 = w + 0 = w \neq 0 = T(x \oplus_V v)$.  Thus $T$ does not preserve addition, and hence is not a linear map. $\qedsymbol$

\section*{Sec. 3B}
\subsubsection*{\#3}
{\it Suppose $(v_1, \dots v_m)$ is a list of vectors in $V$.  Define $T \in \mathcal{L}(\mathbb{F}^m, V)$ by}
\begin{align*}
T(z_1, \dots, z_m) = z_1v_1 + \dots + z_mv_m
\end{align*}
\begin{enumerate}[\ \ \ \bf (a)\ ]
\item {\it What property of $T$ corresponds to $(v_1, \dots, v_m)$ spanning $V$?} $\boxed{\text{Surjectivity}}$ \\

Suppose $(v_1, \dots, v_m)$ spans V.  Then $\forall v \in V$, $\exists (z_1, \dots, z_m) \in \mathbb{F}^m$ such that $z_1v_1 + \dots + z_mv_m = v$.  Then $\exists z \in \mathbb{F}^m$ such that $T(z) = v$.  Thus $T$ is surjective.  Now suppose $T$ is surjective.  Then $\forall v \in V$, $\exists z \in \mathbb{F}^m$ such that $T(z) = v$.  Since $z \in \mathbb{F}^m$, $z = (z_1, \dots, z_m)$ for some $z_1, \dots, z_m \in \mathbb{F}$.  Then $v$ = $z_1v_1 + \dots z_mv_m$.  Since $v$ was arbitrary in $V$, $(v_1, \dots, v_m)$ spans $V$.
\item {\it What property of $T$ corresponds to $(v_1, \dots, v_m)$ being linearly independent?} $\boxed{\text{Injectivity}}$ \\

Suppose $(v_1, \dots, v_m)$ is linearly independent.  Then $\forall v \in \text{span}(v_1, \dots, v_m)$, \\$\exists! (z_1, \dots, z_m) \in \mathbb{F}^m$ such that $v = z_1v_1 + \dots + z_mv_m$.  Then suppose $z, y \in \mathbb{F}^m$ (so $z = (z_1, \dots, z_m)$ and $y = (y_1, \dots, y_m)$ for some $z_1, \dots, z_m, y_1, \dots, y_m \in \mathbb{F}$), and suppose $T(z) = T(y)$.  Then $z_1v_1 + \dots + z_mv_m = y_1v_1 + \dots + y_mv_m \implies (z_1 - y_1)v_1 + \dots + (z_m - y_m)v_m = 0$.  Thus $z_1 = y_1$, \dots, $z_m = y_m \implies z = y$.  Thus $T$ is injective.  Now suppose $T$ is injective.  Then $T(z) = T(y) \implies z = y$.  Then let $z = (z_1, \dots, z_m)$ and $z_1v_1 + \dots + z_mv_m = 0$.  But $0 = 0v_1 + \dots + 0v_m$.  Then since $T$ is injective, $z_1 = \dots = z_m = 0$.  Thus $(v_1, \dots, v_m)$ is linearly independent.

\end{enumerate}
\subsubsection*{\#5}
{\it Give an example of a linear map $T: \mathbb{R}^4 \rightarrow \mathbb{R}^4$ such that}
\begin{align*}
\text{range }T = \text{null }T
\end{align*}
Define $T: \mathbb{R}^4 \rightarrow \mathbb{R}^4$ by $T(w, x, y, z) = (y, z, 0, 0)$.  Since $y$ and $z$ are arbitrary in $\mathbb{R}$,
\begin{align*}
\text{range }T = \{(a, b, 0, 0) \in \mathbb{R}^4 \suchthat a, b, \in \mathbb{R}\}
\end{align*}
Now suppose $T(w, x, y, z) = 0_{\mathbb{R}^4} = (0, 0, 0, 0)$.  Then $(y, z, 0, 0) = (0, 0, 0, 0) \implies y = z = 0$.  Since $w$ and $x$ are arbitrary in $\mathbb{R}^4$,
\begin{align*}
\text{null }T = \{(a, b, 0, 0) \in \mathbb{R}^4 \suchthat a, b, \in \mathbb{R}\} = \text{range } T
\end{align*}

\subsubsection*{\#6}
{\it Prove that there does not exist a linear map $T: \mathbb{R}^5 \rightarrow \mathbb{R}^5$ such that}
\begin{align*}
\text{range }T = \text{null }T
\end{align*}
Let $T \in \mathcal{L}(\mathbb{R}^5, \mathbb{R}^5)$.  Then $\text{dim }\mathbb{R}^5 = \text{dim }\text{range }T + \text{dim }\text{null }T$.  Suppose $ \text{range }T = \text{null }T$.  Then $\text{dim }\text{range }T = \text{dim }\text{null }T$.  Thus
\begin{align*}
2(\text{dim }\text{range }T) &= \text{dim }\mathbb{R}^5 = 5 \\
\implies \text{dim }\text{range }T &= \frac{5}{2}
\end{align*}
This is a contradiction since the dimension of a vector space must be a natural number.  Thus there does not exist a linear map $T: \mathbb{R}^5 \rightarrow \mathbb{R}^5$ such that $\text{range }T = \text{null }T$. $\qedsymbol$

%\pagebreak
%\begin{thebibliography}{99}
%
%\bibitem{Abrams1997b}
%Abrams, P.~A. and Matsuda, H.
%Prey Adaptation as a Cause of Predator-Prey Cycles.
%\emph{Evolution}
%1997, 51:1742-1750.
%
%\bibitem{Chavez2001}
%Brauer, F., Castillo-Chavez, C.
%Mathematical Models in Population Biology and Epidemiology.
%Springer,
%2011. Print.
%
%\bibitem{Boyce2012}
%Boyce, W. E., and DiPrima, R. C.
%Elementary Differential Equations and Boundary Value Problems %10\textsuperscript{th} ed.
%Wiley Global Education
%2012. Print.
%
%\bibitem{Saloniemi1993}
%Saloniemi, I.
%A Coevolutionary Predator-Prey Model with Quantitative Characters.
%\emph{American Naturalist}
%1993, 141:880-896.
%
%\bibitem{Schreiber2011}
%Schreiber, S.~J., B$\ddot{\mbox{u}}$rger,  R., and Bolnick,  D.~I.
%The Community Effects of Phenotypic and Genetic Variation within a Predator %Population.
%\emph{Ecology}
%2011,  92(8):526-543. 
%
%\end{thebibliography}

\end{document}