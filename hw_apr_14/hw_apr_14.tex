\documentclass[12pt]{article}
\textwidth=17cm \oddsidemargin=-0.9cm \evensidemargin=-0.9cm
\textheight=23.7cm \topmargin=-1.7cm

\usepackage{amssymb, amsmath, amsfonts}
\usepackage{moreverb}
\usepackage{graphicx}
\usepackage{enumerate}
\usepackage{graphics}
\usepackage{color}
\usepackage{array}
\usepackage{float}
\usepackage{hyperref}
\usepackage{textcomp}
\usepackage{alltt}
\usepackage{mathtools}
\usepackage{tikz}
\usetikzlibrary{positioning}
\usetikzlibrary{arrows}
\usepackage{pgfplots}
\usepackage{bigints}
\allowdisplaybreaks

\newcommand{\suchthat}{\, \mid \,}
\renewcommand{\theenumi}{\alph{enumi}}
\newcommand\Wider[2][3em]{%
\makebox[\linewidth][c]{%
  \begin{minipage}{\dimexpr\textwidth+#1\relax}
  \raggedright#2
  \end{minipage}%
  }%
}

\setcounter{section}{-1}

\title{\bf HW: Custom Problems, Section 8A \#7, Section 8B \#1, 5, Section 8C \#1, 3}
\author{\bf Sam Fleischer}
\date{\bf April 14, 2015}

\begin{document}
{\bf MATH 462 \hfill Advanced Linear Algebra \ \ \ \ \ \hfill Spring 2015} 

{\let\newpage\relax\maketitle}

\section*{Custom Problems}
\subsection*{\#1}
{\it Let $U=\{(z_1, z_2, z_3) \in \mathbb{C}^3 \suchthat z_1 + z_2 + z_3 = 0\}$, and let $f_1 = (1, -1, 0)$, and $f_2 = (1, 0, -1)$.  Define $T\in \mathcal{L}(U)$ by $T(f_1) = (1, 1, -2)$ and $T(f_2) = f_2$.}
\begin{enumerate}[\ \ \it(a)\ \ ]
	\item {\it Find the eigenvalues and associated eigenspaces for $T$.} \\

	\noindent First note that $(f_1, f_2)$ is a basis of $U$ since they are linearly independent and $\text{dim}(U) = 2$.  Thus for any $v \in U$, $v = a_1f_1 + a_2f_2$.  To find the eigenvalues and eigenvectors of $T$, set $T(v) = \lambda v$.
	\begin{align*}
		&T(v) = \lambda v \\
		\implies &T(a_1f_1 + a_2f_2) = \lambda(a_1f_1 + a_2f_2) \\
		\implies &a_1(1, 1, -2) + a_2(1, 0, -1) = \lambda(a_1(1, -1, 0) + a_2(1, 0, -1)) \\
		\implies &(a_1 + a_2, a_1, -2a_1 - a_2) = (\lambda(a_1 + a_2), -\lambda a_1, -\lambda a_2) \\
		\implies &\left\{\begin{array}{rl}
			a_1 + a_2 & = \lambda(a_1 + a_2) \\
			a_1 & = -\lambda a_1 \\
			-2a_1 - a_2 & = -\lambda a_2
		\end{array}\right.
	\end{align*}
	$a_1 = -\lambda a_1 \implies a_1 = 0$ or $\lambda = -1$.  If $a_1 = 0$, then $a_1 + a_2 = \lambda(a_1 + a_2) \implies a_2 = 0$ or $\lambda = 1$.  If $a_2 = 0$, then $v = 0$, and thus $\lambda$ would not be an eigenvalue.  Thus $\lambda = 1$.  Thus $-2a_1 - a_2 = \lambda a_2 \implies a_2$ is arbitrary.  Thus $\lambda = 1$ is an eigenvalue of $T$ and $E(1) = \{a(1, 0, -1) \suchthat a \in \mathbb{C}\}$.  Now suppose $\lambda = -1$.  Thus $a_1 + a_2 = 0$.  So $-1$ is an eigenvalue of $T$ and $E(-1) = \{a_1f_1 + a_2f_2 \in U \suchthat a_1 + a_2 = 0\}$.  However, since $a_1 = -a_2$, $E(-1) = \{a(0, -1, 1) \suchthat a \in \mathbb{C}\}$.
	\item {\it Determine whether there exists a basis for $U$ so that $\mathcal{M}(T)$ is diagonal with respect to that basis.  If so, find suh a basis, compute $\mathcal{M}(T)$ with respect to that basis, and ignore (c) and (d).  If not, move on to parts (c) and (d).} \\

	\noindent Note $\text{dim}(E(-1)) = 1$ and $\text{dim}(E(1)) = 1$.  Let $b_1 = (1, 0, -1) \in E(1)$ and $b_2 = (0, -1, 1) \in E(-1)$.  Then $\pi = (b_1, b_2)$ is linearly independent since vectors from different eigenspaces are linearly independent.  Since $\text{dim}(U) = 2 = \text{len}(\pi)$, $\pi$ is a basis for $U$ comprised entirely of eigenvectors.  Thus $\mathcal{M}(T, \pi)$ is a diagonal matrix.
	\begin{align*}
		b_1 &= (1, 0, -1) = f_2 \\
		\implies T(b_1) &= T(f_2) = f_2 = b_1 \\
		b_2 &= (0, -1, 1) = (1, -1, 0) - (1, 0, -1)  = f_1 - f_2 \\
		\implies T(b_2) &= T(f_1) - T(f_2) = (1, 1, -2) - (1, 0, -1) = (0, 1, -1) = -b_2 \\
		\implies \mathcal{M}(T, \pi) &= \left(\begin{array}{cc}
			1 & 0 \\
			0 & -1
		\end{array}\right)
	\end{align*}
	We can ignore {\it(c)} and {\it(d)}.
	\item {\it Compute the generalized eigenspaces for $T$, and find a basis for $U$ consisting of generalized eigenvectors.}

	\noindent (((IGNORE)))
	\item {\it Compute $\mathcal{M}(T)$ with respect to the basis that is your answer in part (c), arranging your answer in block diagonal form.}

	\noindent (((IGNORE)))
\end{enumerate}

\subsection*{\#2}
{Let $T \in \mathcal{L}(\mathbb{C}^3$ by $T(z_1, z_2, z_3) = (4z_1 + 4z_2 + 4z_3, 5z_2 + 4z_3, 5z_3)$}
\begin{enumerate}[\ \ \it(a)\ \ ]
	\item {\it Find the eigenvalues and associated eigenspaces for $T$.} \\

	\noindent Let $e = (e_1, e_2, e_3)$ be the standard basis of $\mathbb{C}^3$.  Then
	\begin{align*}
		T(e_1) &= (4, 0, 0) \\
		T(e_2) &= (4, 5, 0) \\
		T(e_3) &= (4, 4, 5) \\
		\implies \mathcal{M}(T, e) &= \left(\begin{array}{ccc}
		4 & 4 & 4 \\
		0 & 5 & 4 \\
		0 & 0 & 5
		\end{array}\right)
	\end{align*}
	Since $\mathcal{M}(T, e)$ is an upper-triangular matrix, the eigenvalues of $T$ are the entries on the main diagonal, namely $\lambda = 4$ and $\lambda = 5$.  To calculate $E(\lambda)$, find the conditions on $v$ by which $T(z_1, z_2, z_3) = \lambda (z_1, z_2, z_3)$.
	\begin{align*}
		T(z_1, z_2, z_3) &= 4(z_1, z_2, z_3) \\
		\implies (4z_1 + 4z_2 + 4z_3, 5z_2 + 4z_3, 5z_3) &= (4z_1, 4z_2, 4z_3) \\
		&\implies \left\{\begin{array}{rl}
			4z_1 + 4z_2 + 4z_3 &= 4z_1 \\
			5z_2 + 4z_3 &= 4z_2 \\
			5z_3 &= 4z_3 \\
		\end{array}\right. \\
		&\implies z_3 = 0 \\
		&\implies z_2 = 0 \\
		&\implies z_1 \text{ is arbitrary} \\
		&\implies E(4) = \{a(1, 0, 0) \suchthat a \in \mathbb{C}\}
	\end{align*}
	\begin{align*}
		T(z_1, z_2, z_3) &= 5(z_1, z_2, z_3) \\
		\implies (4z_1 + 4z_2 + 4z_3, 5z_2 + 4z_3, 5z_3) &= (5z_1, 5z_2, 5z_3) \\
		&\implies \left\{\begin{array}{rl}
			4z_1 + 4z_2 + 4z_3 &= 5z_1 \\
			5z_2 + 4z_3 &= 5z_2 \\
			5z_3 &= 5z_3 \\
		\end{array}\right. \\
		&\implies z_3 = 0 \\
		&\implies z_1 = 4z_2 \text{ , } z_2 \text{ arbitrary} \\
		&\implies E(5) = \{a(4, 1, 0) \suchthat a \in \mathbb{C}\}
	\end{align*}
 	\item {\it Determine whether there exists a basis for $U$ so that $\mathcal{M}(T)$ is diagonal with respect to that basis.  If so, find suh a basis, compute $\mathcal{M}(T)$ with respect to that basis, and ignore (c) and (d).  If not, move on to parts (c) and (d).} \\

	\noindent Note $\text{dim}(E(4)) = 1$ and $\text{dim}(E(5)) = 1$.  Thus each of these eigenspaces may contribute only one vector to a linearly independent list.  Since $\text{dim}(\mathbb{C}^3) = 3$, the by the Pigeonhole Principle, we cannot form basis for $\mathbb{C}^3$ comprised entirely of vectors from $E(4)$ and $E(5)$.  Thus we cannot form a basis comprised entirely of eigenvectors, and thus $T$ is not diagonalizable.  We unfortunately cannot ignore {\it(c)} and {\it(d)}.
	\item {\it Compute the generalized eigenspaces for $T$, and find a basis for $U$ consisting of generalized eigenvectors.}
	
	\noindent By definition, $G(4) = \text{null}(T - 4I)^3$
	\begin{align*}
		\mathcal{M}((T - 4I), e) &= \left(\begin{array}{ccc}
			0 & 4 & 4 \\
			0 & 1 & 4 \\
			0 & 0 & 1
		\end{array}\right) \\
		\implies [\mathcal{M}((T - 4I), e)]^2 &= \left(\begin{array}{ccc}
			0 & 4 & 4 \\
			0 & 1 & 4 \\
			0 & 0 & 1
		\end{array}\right) \cdot \left(\begin{array}{ccc}
			0 & 4 & 4 \\
			0 & 1 & 4 \\
			0 & 0 & 1
		\end{array}\right) = \left(\begin{array}{ccc}
			0 & 4 & 20 \\
			0 & 1 & 8 \\
			0 & 0 & 1
		\end{array}\right) \\
		\implies [\mathcal{M}((T - 4I), e)]^3 &= \left(\begin{array}{ccc}
			0 & 4 & 20 \\
			0 & 1 & 8 \\
			0 & 0 & 1
		\end{array}\right) \cdot \left(\begin{array}{ccc}
			0 & 4 & 4 \\
			0 & 1 & 4 \\
			0 & 0 & 1
		\end{array}\right) = \left(\begin{array}{ccc}
			0 & 4 & 36 \\
			0 & 1 & 12 \\
			0 & 0 & 1
		\end{array}\right) \\
		\implies (T - 4I)^3(e_1) &= 0 \\
		(T - 4I)^3(e_2) &= 4e_1 + e_2 \\
		(T - 4I)^3(e_3) &= 36e_1 + 12e_2 + e_3 \\
		\implies (T - 4I)^3(z_1, z_2, z_3) &= z_2(4e_1 + e_2) + z_3(36e_1 + 12e_2 + e_3) \\
		&= (4z_2 + 36z_3, z_2 + 12z_3, z_3)
	\end{align*}
	So $(T - 4I)^3(z_1, z_2, z_3) = 0$ only if $z_2 = z_3 = 0$.  Thus,
	\begin{align*}
		G(4) = \{a(1, 0, 0) \suchthat a \in \mathbb{C}\} = E(4)
	\end{align*}
	\noindent By definition, $G(5) = \text{null}(T - 5I)^3$
	\begin{align*}
		\mathcal{M}((T - 5I), e) &= \left(\begin{array}{ccc}
			-1 & 4 & 4 \\
			0 & 0 & 4 \\
			0 & 0 & 0
		\end{array}\right) \\
		\implies [\mathcal{M}((T - 4I), e)]^2 &= \left(\begin{array}{ccc}
			-1 & 4 & 4 \\
			0 & 0 & 4 \\
			0 & 0 & 0
		\end{array}\right) \cdot \left(\begin{array}{ccc}
			-1 & 4 & 4 \\
			0 & 0 & 4 \\
			0 & 0 & 0
		\end{array}\right) = \left(\begin{array}{ccc}
			1 & -4 & 12 \\
			0 & 0 & 0 \\
			0 & 0 & 0
		\end{array}\right) \\
		\implies [\mathcal{M}((T - 4I), e)]^3 &= \left(\begin{array}{ccc}
			1 & -4 & 12 \\
			0 & 0 & 0 \\
			0 & 0 & 0
		\end{array}\right) \cdot \left(\begin{array}{ccc}
			-1 & 4 & 4 \\
			0 & 0 & 4 \\
			0 & 0 & 0
		\end{array}\right) = \left(\begin{array}{ccc}
			-1 & 4 & -12 \\
			0 & 0 & 0 \\
			0 & 0 & 0
		\end{array}\right) \\
		\implies (T - 5I)^3(e_1) &= -e_1 \\
		(T - 4I)^3(e_2) &= 4e_1 \\
		(T - 4I)^3(e_3) &= -12e_1 \\
		\implies (T - 4I)^3(z_1, z_2, z_3) &= z_1(-e_1) + z_2(4e_1) + z_3(-12e_1) \\
		&= (-z_1 + 4z_2 -12z_3, 0, 0)
	\end{align*}
	So $(T - 5I)^3(z_1, z_2, z_3) = 0$ only if $z_1 = 4z_2 - 12z_3$.  Thus,
	\begin{align*}
		G(5) = \{(4a - 12b, a, b) \suchthat a,b \in \mathbb{C}\} \supset E(5)
	\end{align*}
	$\text{dim}(G(4)) = 1 \implies G(4)$ may contribute only one vector to a linearly independent list.  $\text{dim}(G(5)) = 2 \implies G(5)$ may contribute up to two vectors to a linearly independent list.  Since vectors from different generalized eigenspaces are linearly independent and $\text{dim}(\mathbb{C}^3) = 3$, we can form a basis comprised of one vector from $G(4)$ and two from $G(5)$.  Pick $(1, 0, 0) \in G(4)$, and $(4, 1, 0), (-12, 0, 1) \in G(5)$, and let $\pi = \Big((1, 0, 0), (4, 1, 0), (-12, 0, 1)\Big)$.  Then $\pi$ is a basis of $\mathbb{C}^3$ comprised entirely of generalized eigenvectors.
	\item {\it Compute $\mathcal{M}(T)$ with respect to the basis that is your answer in part (c), arranging your answer in block diagonal form.}

	\begin{align*}
		T(1, 0, 0) &= T(e_1) = 4e_1 = (4, 0, 0) = 4(1, 0, 0) \\
		T(4, 1, 0) &= 4T(e_1) + T(e_2) = 4(4e_1) + (4e_1 + 5e_2) \\
		&= 20e_1 + 5e_2 = (20, 5, 0) = 5(4, 1, 0) \\
		T(-12, 0, 1) &= -12T(e_1) + T(e_3) = -12(4e_1) + (4e_1 + 4e_2 + 5e_3) \\
		&= -44e_1 + 4e_2 + 5e_3 = (-44, 4, 5) = 4(4, 1, 0) + 5(-12, 0, 1) \\
		\implies \mathcal{M}(T, \pi) &= \left(\begin{array}{ccc}
			4 & 0 & 0 \\
			0 & 5 & 4 \\
			0 & 0 & 5
		\end{array}\right) = \left(\begin{array}{cc}
			D_1 & 0 \\
			0 & D_2
		\end{array}\right)
	\end{align*}
	where
	\begin{align*}
		D_1 = \left(\begin{array}{c}
			4
		\end{array}\right) \ \ \ \ \ \text{and} \ \ \ \ \ \hfill D_2 = \left(\begin{array}{cc}
			5 & 4 \\
			0 & 5
		\end{array}\right)
	\end{align*}
\end{enumerate}

\section*{8A}
\subsection*{\#7}
{\it Suppose $N \in \mathcal{L}(V)$ is nilpotent.  Prove that $0$ is the only eigenvalue of $N$.} \\

\noindent Let $\text{dim}(V) = n$.
\begin{align*}
	&N \text{ is nilpotent} \\
	\implies &N^n = \mathbf{0} \\
	\implies &(N - 0I)^n = \mathbf{0} \\
	\implies &\text{null}(N - 0I)^n = V \\
	\implies &G(0) = V.
\end{align*}
Now suppose $\lambda \neq 0$ and $\lambda$ is an eigenvalue of $N$.  Then $G(\lambda)$ is non-trivial, i.e. $G(\lambda) \neq \{0\}$.  Suppose $\pi = (v_1, \dots, v_n)$ is a basis for $V$.  Then $\pi$ is a basis for $G(0)$.  Then pick $w \in G(\lambda)$ such that $w \neq 0$.  Since vectors from different generalized eigenspaces are linearly independent, $\phi = (v_1, \dots, v_n, w)$ is a linearly independent list.  But $\text{len}(\phi) = n+1 > n = \text{len}(\pi)$.  This is a contradiction since a linearly independent set cannot have longer length than a basis.  Thus there is no eigenvalue of $N$ other than $0$. \hfill $\square$

\section*{8B}
\subsection*{\#1}
{\it Suppose $V$ is a complex vector space, $N \in \mathcal{L}(V)$, and $0$ is the only eigenvalue of $N$.  Prove that $N$ is nilpotent.} \\

\noindent Since $0$ is the only eigenvector of $N$, $G(0)$ is the only generalized eigenspace of $N$.  Since every complex vector space has a basis consisting entirely of generalized eigenvectors, $V$ must have a basis comprised entirely of vectors from $G(0)$.  Thus $\text{dim}(G(0)) \geq \text{dim}(V)$.  But $G(0) \subset V$, so $\text{dim}(G(0)) \leq \text{dim}(V)$.  Thus $\text{dim}(G(0)) = \text{dim}(V)$, which implies $G(0) = V$.
\begin{align*}
	G(0) &= V \\
	\implies &\text{null}(N - 0I)^n = V \\
	\implies &(N - 0I)^n = \mathbf{0} \\
	\implies &N^n = \mathbf{0} \\
	\implies &N \text{ is nilpotent}
\end{align*}
\hfill $\square$

\subsection*{\#5}
{\it Suppose $V$ is a complex vector space and $T \in \mathcal{L}(V)$.  Prove that $V$ has a basis consisting of eigenvectors of $T$ if and only if every generalized eigenvector of $T$ is an eigenvector of $T$.} \\

\noindent ``$\Longrightarrow$'' \\
Suppose $V$ has a basis consisting of eigenvectors of $T$.  Let $\pi$ be one such basis, listed in such a way so that eigenvectors from the same eigenspace are adjacent.  Then $T$ is diagonalizable.  In particular, $\mathcal{M}(T, \pi)$ is a diagonal matrix.
\begin{align*}
	\mathcal{M}(T, \pi) &= \left(\begin{array}{ccc}
		A_1 & & 0 \\
		 & \ddots & \\
		0 & & A_m
	\end{array}\right)
\end{align*}
where $A_i = \lambda_iI_{d_i}$ and $d_i = \text{dim}(E(\lambda_i))$, $i = 1, \dots, m$.  However, this is also a block diagonal matrix.  {\color{red}something something proof}.\\

\noindent ``$\Longleftarrow$'' \\
Suppose every generalized eigenvector of $T$ is an eigenvector of $T$.  Since $V$ is a complex vector space, $\exists \pi$ such that $\pi$ is a basis of $V$ consisting of generalized eigenvectors of $T$.  Since every generalized eigenvector is an eigenvector, $\pi$ is consisting of eigenvectors of $T$.  Thus $V$ has a basis of consisting of eigenvectors of $T$.\\

\noindent Thus, $V$ has a basis consisting of eigenvectors of $T$ if and only if every generalized eigenvector of $T$ is an eigenvector of $T$. \hfill $\square$

\section*{8C}
\subsection*{\#1}
{\it Suppose that $T \in \mathcal{L}(\mathbb{C}^4)$ is such that the eigenvalues of $T$ are $3$, $5$, and $8$.  Prove that $(T - 3I)^2(T - 5I)^2(T - 8I)^2 = \mathbf{0}$.} \\

\noindent Since the eigenvalues of $T$ are $3$, $5$, and $8$, the characteristic polynomial of $T$ is $(T - 3I)^{d_3}(T - 5I)^{d_5}(T - 8I)^{d_8}$ where $d_\lambda$ is the multiplicity of eigenvalue $\lambda$.  Since $\text{dim}(\mathbb{C}^4) = 4$, $d_3 + d_5 + d_8 = 4$.  But $d_\lambda \geq 1$ for each eigenvalue.  Thus $d_\lambda = 1$ for two of the three eigenvalues, and $d_\lambda = 2$ for exactly one of the three eigenvalues.  Without loss of generality, suppose $d_3 = 2$, and $d_5 = d_8 = 1$.  Then the characteristic polynomial is $p = (z - 3)^2(z - 5)(z - 8)$.  Then $q = (z - 3)^2(z - 5)^2(z - 8)^2$ is a polynomial multiple of the $p$.  However, $p$ is a polynomial multiple of the minimal polynomial $m$, and thus $q$ is a polynomial multiple of $m$.  Thus $q(T) = (T - 3I)^2(T - 5I)^2(T - 8I)^2 = \mathbf{0}$. \hfill $\square$

\subsection*{\#3}
{\it Give an example of an operator of $\mathbb{C}^4$ whose characteristic polynomial equals $(z - 7)^2(z - 8)^2$.} \\

\noindent {\color{red} SOMETHING}

\end{document}
