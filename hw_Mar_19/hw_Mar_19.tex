\documentclass[12pt]{article}
\textwidth=17cm \oddsidemargin=-0.9cm \evensidemargin=-0.9cm
\textheight=23.7cm \topmargin=-1.7cm

\usepackage{amssymb, amsmath, amsfonts}
\usepackage{moreverb}
\usepackage{graphicx}
\usepackage{enumerate}
\usepackage{graphics}
\usepackage{color}
\usepackage{array}
\usepackage{float}
\usepackage{hyperref}
\usepackage{textcomp}
\usepackage{alltt}
\usepackage{mathtools}
\usepackage{tikz}
\usetikzlibrary{positioning}
\usetikzlibrary{arrows}
\usepackage{pgfplots}
\usepackage{bigints}

\newcommand{\suchthat}{\, \mid \,}
\renewcommand{\theenumi}{\alph{enumi}}
\newcommand\Wider[2][3em]{%
\makebox[\linewidth][c]{%
  \begin{minipage}{\dimexpr\textwidth+#1\relax}
  \raggedright#2
  \end{minipage}%
  }%
}

\setcounter{section}{-1}

\title{\bf HW: Custom Problem, Section 5A \#19, 21, Section 5C \#14, 15}
\author{\bf Blam Fleischer}
\date{\bf March 19, 2015}

\begin{document}
{\bf MATH 462 \hfill Advanced Linear Algebra \ \ \ \ \ \hfill Spring 2015} 

{\let\newpage\relax\maketitle}

\section*{Custom Problem}
{\it Let $T \in \mathcal{L}(\mathbb{C}^2)$ by $T(z, w) = (2w, -8z)$.}
	\begin{enumerate}[\it\ \ (a)\ \ ]
		\item {\it Find a basis for $\mathbb{C}^2$ consisting of eigenvectors of $T$.}\\
		
		\noindent To find the eigenvalues, set $T(z, w) = \lambda(z, w)$.
		\begin{align*}
			T(z, w) &= \lambda(z, w) \\
			\implies (2w, -8z) &= (\lambda z, \lambda w) \\
			\implies 2w &= \lambda z\text{,} \ \ \ \ \text{and}\ \ \ \ -8z = \lambda w \\
			\implies -8z &= \frac{\lambda^2}{2}z
		\end{align*}
		If $z = 0$, then $w = 0$.  Then $\lambda$ would not be an eigenvalue.  So we suppose $z \neq 0$, and thus
		\begin{align*}
			-8 &= \frac{\lambda^2}{2} \\
			\implies -16 &= \lambda^2 \\
			\implies \lambda &= \pm 4i
		\end{align*}
		Thus the two eigenvalues for $T$ are $4i$ and $-4i$.  If $\lambda = 4i$, then $2w = \lambda z \implies w = 2iz \implies E(4i) = \{(z, 2iz) \in \mathbb{C}^2 \suchthat z \in \mathbb{C}\}$.  Specifically, $(1, 2i) \in E(4i)$.  If $\lambda = -4i$, then $2w = \lambda z \implies w = -2iz \implies E(-4i) = \{(z, -2iz) \in \mathbb{C}^2 \suchthat z \in \mathbb{C}\}$.  Specifically, $(1, -2i) \in E(-4i)$.  Then $\pi = ((1, 2i), (1, -2i))$ is a linearly independent list since it is comprised entirely of eigenvectors corresponding to distinct eigenvalues.  Since $\text{dim}(\mathbb{C}^2) = 2 = \text{len}(\pi)$, then $\pi$ is a basis for $\mathbb{C}^2$ consisting of eigenvectors of $T$.
		\item {\it Find $\mathcal{M}(T)$ with respect to this basis.}\\
		
		\noindent Since
		\begin{align*}
			T(1, 2i) &= (2(2i), -8(1)) \\
			&= (4i, -8) \\
			&= 4i(1, 2i) + 0(1, -2i)
		\end{align*}
		and
		\begin{align*}
		T(1, -2i) &= (2(-2i), -8(1)) \\
		&= (-4i, -8) \\
		&= 0(1, 2i) -4i(1, -2i)
		\end{align*}
		then 
		\begin{align*}
			\mathcal{M}(T, \pi) = \left(\begin{array}{cc}
				4i & 0 \\
				0 & -4i
			\end{array}\right)
		\end{align*}
	\end{enumerate}
	
\section*{5A}
\subsection*{\#19}
	{\it Suppose $n$ is a positive integer and $T \in \mathcal{L}(\mathbb{F}^n)$ is defined by}
	\begin{align*}
		T(x_1, \dots, x_n) = (x_1 + \dots + x_n, \dots, x_1 + \dots + x_n)
	\end{align*}
	{\it in other words, $T$ is the operator whose matrix (with respect to the standard basis) consists of all $1$'s.  Find the eigenvalues and eigenvectors of $T$.}\\
	
	\noindent To find the eigenvalues, set $T(x_1, \dots, x_n) = \lambda(x_1, \dots, x_n)$.
	\begin{align*}
		T(x_1, \dots, x_n) &= \lambda(x_1, \dots, x_n) \\
		\implies (x_1 + \dots + x_n, \dots, x_1 + \dots + x_n) &= (\lambda x_1, \dots, \lambda x_n) \\
		\implies x_1 + \dots + x_n &= \lambda x_1 = \dots = \lambda x_n\\
		\implies \lambda &= \frac{x_1 + \dots + x_n}{x_1} = \dots = \frac{x_1 + \dots + x_n}{x_n}
	\end{align*}
	If $\lambda \neq 0$, then $x_1 = \dots = x_n$.  This implies $\lambda = n x_n$ is an eigenvalue and $((1, \dots, 1))$ is a basis for $E(n x_n) = \{(x, \dots, x) \in \mathbb{F}^n \suchthat x \in \mathbb{F}\}$.  However, if $\lambda = 0$, then the equation is true only if $x_1 + \dots + x_n = 0$.  Thus $0$ is an eigenvalue and $E(0) = \{(x_1, \dots, x_n) \in \mathbb{F}^n \suchthat x_1 + \dots + x_n = 0\}$.  Note $\text{dim}(E(0)) = n-1$ and a basis for $E(0)$ is
	\begin{align*}
		\pi = ((1, -1, 0, \dots, 0), (1, 0, -1, 0, \dots, 0), \dots, (1, 0, \dots, 0, -1))
	\end{align*}
	Let $\pi'$ be the concatenation of $(1, \dots, 1)$ and $\pi$.  In other words,
	\begin{align*}
		\pi' = ((1, \dots, 1),(1, -1, 0, \dots, 0), (1, 0, -1, 0, \dots, 0), \dots, (1, 0, \dots, 0, -1))
	\end{align*}
	Since $\pi$ is a basis for $E(0)$, then $\pi$ is a linearly independent list.  However, $((1, \dots, 1))$ is a basis for $E(n x_n)$ and any two vectors from different eigenspaces are linearly independent.  Thus $\pi'$ is a linearly independent list.  Note
	\begin{align*}
		\text{dim}(\mathbb{F}^n) = n = 1 + (n-1) = \text{dim}(E(n x_n)) + \text{dim}(E(0)) = \text{len}(\pi')
	\end{align*}
	Thus $\pi'$ is a basis for $\mathbb{F}^n$.  Note the following:
	\begin{align*}
		T(1, \dots, 1) &= (n, \dots, n) = n(1, \dots, 1)\\
		T(1, -1, 0, \dots, 0) &= (0, \dots, 0) \\
		T(1, 0, -1, 0, \dots, 0) &= (0, \dots, 0) \\
		&\vdots \\
		(1, 0, \dots, 0, -1) &= (0, \dots, 0)
	\end{align*}
	Thus,
	\begin{align*}
		\mathcal{M}(T, \pi') = \left(\begin{array}{cccc}
			n & 0 & \dots & 0 \\
			0 & 0 & & \vdots \\
			\vdots & & \ddots & \\
			0 & \dots & & 0
			\end{array}\right)
	\end{align*}
\subsection*{\#21}
{\it Suppose $T \in \mathcal{L}(V)$ is invertible.}
	\begin{enumerate}[\it\ \ (a)\ \ ]
		\item {\it Suppose $\lambda \in \mathbb{F}$ with $\lambda \neq 0$.  Prove that $\lambda$ is an eigenvalue of $T$ if and only if $\dfrac{1}{\lambda}$ is an eigenvalue of $T^{-1}$.} \\
		\begin{align*}
			&\lambda \text{ is an eigenvalue of }T \\[.1cm]
			\iff &\exists v \in V \text{such that }T(v) = \lambda v \\[.1cm]
			\iff &T^{-1}(T(v)) = T^{-1}(\lambda v) \\[.1cm]
			\iff &v = \lambda T^{-1}(v) \\[.1cm]
			\iff &T^{-1}(v) = \frac{1}{\lambda}v \text{ for some $v \in V$}\\[.1cm]
			\iff &\frac{1}{\lambda} \text{ is an eigenvalue of }T^{-1}
		\end{align*}\hfill $\square$
		\item {\it Prove that $T$ and $T^{-1}$ have the same eigenvectors.} \\
		
		\noindent Let $\hat{v}$ be an eigenvector of $T$ corresponding to an arbitrary eigenvalue of $T$'s, say $\hat{\lambda}$.  Then
		\begin{align*}
			T(\hat{v}) &= \hat{\lambda}\hat{v} \\
			\implies T^{-1}(T(\hat{v})) &= T^{-1}(\hat{\lambda}\hat{v}) \\
			\implies \hat{v} &= \hat{\lambda}T^{-1}(\hat{v}) \\
			\implies T^{-1}(\hat{v}) &= \frac{1}{\hat{\lambda}}\hat{v} \\
		\end{align*}
		Thus $\hat{v}$ is an eigenvector of $T^{-1}$ corresponding to $\dfrac{1}{\hat{\lambda}}$.  Thus any eigenvector of $T$ is an eigenvector of $T^{-1}$.  However, $(T^{-1})^{-1} = T$ implies any eigenvector of $T^{-1}$ is an eigenvector of $T$.  Thus $T$ and $T^{-1}$ have the same eigenvectors. \hfill $\square$
	\end{enumerate}
\section*{5C}
\subsection*{\#14}
{\it Find $T \in \mathcal{L}(\mathbb{C}^3)$ such that $6$ and $7$ are eigenvalues of $T$ such that $T$ does not have a diagonal matrix with respect to any basis of $\mathbb{C}^3$.} \\

\noindent Define $T \in \mathcal{L}(\mathbb{C}^3)$ by
\begin{align*}
	T(z_1, z_2, z_3) = (6z_1 + 3z_2 + 4z_3, 6z_2 + z_3, 7z_3)
\end{align*}
Then let $e = (e_1, e_2, e_3)$ be the standard basis.
\begin{align*}
	T(e_1) = (6, 0, 0) &= 6e_1 \\
	T(e_2) = (3, 6, 0) &= 3e_1 + 6e_2 \\
	T(e_3) = (4, 1, 7) &= 4e_1 + 1e_2 + 7e_3
\end{align*}
Thus
\begin{align*}
	\mathcal{M}(T, e) = \left(\begin{array}{ccc}
		6 & 3 & 4 \\
		0 & 6 & 1 \\
		0 & 0 & 7
	\end{array}\right)
\end{align*}
Since $\mathcal{M}(T, e)$ is an upper-triangular matrix, the entries on the main diagonal are the eigenvalues of $T$.  Thus $6$ and $7$ are the eigenvalues of $T$.  To find the eigenvectors of $T$ corresponding to $6$, set $T(z_1, z_2, z_3) = 6(z_1, z_2, z_3)$.  In other words, $(6z_1 + 3z_2 + 4z_3, 6z_2 + z_3, 7z_3) = (6z_1, 6z_2, 6z_3)$, or 
\begin{align*}
	\begin{cases}
		6z_1 + 3z_2 + 4z_3 &= 6z_1 \\
		6z_2 + z_3 &= 6z_2 \\
		7z_3 &= 6z_3
	\end{cases}
	\implies\begin{cases}
		z_1 &\text{ is arbitrary in $\mathbb{C}$} \\
		z_2 &= 0 \\
		z_3 &= 0
	\end{cases}
	\implies E(6) = \{(z, 0, 0) \suchthat z \in \mathbb{C}\}
\end{align*}
To find the eigenvectors of $T$ corresponding to $7$, set $T(z_1, z_2, z_3) = 7(z_1, z_2, z_3)$.  In other words, $(6z_1 + 3z_2 + 4z_3, 6z_2 + z_3, 7z_3) = (7z_1, 7z_2, 7z_3)$, or 
\begin{align*}
	\begin{cases}
		6z_1 + 3z_2 + 4z_3 &= 7z_1 \\
		6z_2 + z_3 &= 7z_2 \\
		7z_3 &= 7z_3
	\end{cases}
	\implies\begin{cases}
		z_1 &= 7z_3 \\
		z_2 &= z_3 \\
		z_3 &\text{ is arbitrary in $\mathbb{C}$}
	\end{cases}
	\implies E(7) = \{(7z, z, z) \suchthat z \in \mathbb{C}\}
\end{align*}
Note $\text{dim}(E(7)) = \text{dim}(E(6)) = 1$, and thus we can never form a linearly independent list comprised entirely of eigenvectors of length more than $2$.  Since $\text{dim}(\mathbb{C}^3) = 3$, all bases are lists of length $3$.  Thus there does not exist a basis for $\mathbb{C}^3$ consisting entirely of eigenvectors.  This is equivalent to saying there does not exist a basis for $\mathbb{C}^3$ such that $\mathcal{M}(T)$ with respect to that basis is diagonal. \hfill $\square$

\subsection*{\#15}
{\it Suppose $T \in \mathcal{L}(\mathbb{C}^3)$ such that $6$ and $7$ are eigenvalues of $T$.  Furthermore, suppose $T$ does not have a diagonal matrix with respect to any basis of $\mathbb{C}^3$.  Prove that there exists $(x, y, z) \in \mathbb{F}^3$ such that $T(x, y, z) = (17 + 8x, \sqrt{5} + 8y, 2\pi + 8z)$.}\\

\noindent Note the following:
\begin{align*}
	T(x, y, z) &= (17 + 8x, \sqrt{5} + 8y, 2\pi + 8z) \\
	\iff T(x, y, z) &= (17, \sqrt{5}, 2\pi) + 8(x, y, z) \\
	\iff T(x, y, z) - 8(x, y, z) &= (17, \sqrt{5}, 2\pi) \\
	\iff T(x, y, z) - 8I(x, y, z) &= (17, \sqrt{5}, 2\pi) \\
	\iff (T - 8I)(x, y, z) &= (17, \sqrt{5}, 2\pi)
\end{align*}
It suffices to show $\exists(x, y, z) \in \mathbb{F}^3$ such that $(T - 8I)(x, y, z) = (17, \sqrt{5}, 2\pi)$.  To do this, we will show $8$ is not an eigenvalue, which will imply $T - 8I$ is surjective, proving the result. \\

\noindent Assume $8$ is an eigenvalue.  Then $\exists v_1 \neq 0$ such that $v_1 \in E(8)$.  Similarly, since $6$ and $7$ are eigenvalues, there exist non-zero elements $v_2$ and $v_3$ such that $v_2 \in E(6)$ and $v_3 \in E(7)$.  Let $\tau = (v_1, v_2, v_3)$.  Since a list containing elements from distinct eigenspaces is linearly independent, $\tau$ is linearly independent.  Also, since $\text{len}(\tau) = 3 = \text{dim}(\mathbb{F}^3)$, then $\tau$ is a basis for $\mathbb{F}^3$.  Since $\tau$ consists entirely of eigenvectors, $T$ is diagonalizable $\Longrightarrow\Longleftarrow$.  Thus $8$ is not an eigenvalue.  Thus $(T - 8I)$ is surjective.  Thus $\exists (x, y, z) \in \mathbb{F}^3$ such that $(T - 8I)(x, y, z) = (17, \sqrt{5}, 2\pi)$.  Thus $\exists (x, y, z) \in \mathbb{F}^3$ such that $T(x, y, z) = (17 + 8x, \sqrt{5} + 8y, 2\pi + 8z)$. \hfill $\square$

\end{document}
