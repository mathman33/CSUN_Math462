\documentclass[12pt]{article}

\usepackage{amssymb, amsmath, amsfonts}
\usepackage{amsthm}
\usepackage{moreverb}
\usepackage{graphicx}
\usepackage{enumerate}
\usepackage{graphics}
\usepackage{color}
\usepackage{array}
\usepackage{float}
\usepackage{hyperref}
\usepackage{textcomp}
\usepackage{alltt}
\usepackage{mathtools}
\usepackage[T1]{fontenc}
\usepackage{fullpage}
\usepackage{changepage}
\usepackage{tikz}
\usepackage[utf8]{inputenc}
\newcommand{\suchthat}{\, \mid \,}
\allowdisplaybreaks
\def\arraystretch{1.3}

\begin{document}

{\bf MATH 462 \hfill Advnaced Linear Algebra \hfill Spring 2015}

\title{\bf Homework: Sec. 5B \# 3, 9}
\author{\bf Sam Fleischer}
\date{\bf Due: Tues. Mar. 12, 2015}

{\let\newpage\relax\maketitle}
\maketitle

\section*{Sec. 5B}
\subsection*{\# 3}
{\it Suppose $T \in \mathcal{L}(V)$ and $T^2 = I$ and $-1$ is not an eigenvalue ot $T$.  Prove $T = I$.} \\

\noindent By the multiplicative properties of polynomials and operators, $T^2 - I = (T - I)(T + I)$.  But $T^2 = I \implies T^2 - I = {\bf 0}$ where ${\bf 0}$ is the zero operator.  Thus $(T - I)(T + I)(v) = {\bf 0}(v) = 0\ \forall v \in V$.  Since $-1$ is not an eigenvalue of $T$, $T + I$ is bijective, and thus $\text{range}(T) = V$.  Then $\forall w \in V,\ \exists v \in V$ such that $(T + I)(v) = w$.  Then $(T - I)(w) = 0,\ \forall w \in V$.  Then $T - I = {\bf 0} \implies T = I$. \qed

\subsection*{\# 9}
{\it Suppose $V$ is finite dimensional, $T \in \mathcal{L}(V)$, and $v \in V$ with $v \neq 0$.  Let $p$ be a nonzero polynomial of smallest degree such that $(p(T))(v) = 0$.  Prove that every zero of $p$ is an eigenvalue of $T$.} \\

\noindent Let $\lambda$ be a zero of $p$.  Then $p(x) = (x - \lambda)q(x)$ for some polynomial $q(x)$ with $\text{deg}(q) = \text{deg}(p) - 1$.  Then $p(T) = (T - \lambda I)q(T)$.
\begin{align*}
(p(T))(v) = 0 \implies (T - \lambda I)(q(T))(v) = 0
\end{align*}
Since $p$ is a polynomial of smallest degree such that $(p(T))(v) = 0$ and $\text{deg}(q) < \text{deg}(p)$, then $(q(T))(v) = w$ for some $w\in V$, $w \neq 0$.
\begin{align*}
(T - \lambda I)(q(T))(v) = 0 \implies (T - \lambda I)(w) = 0
\end{align*}
Thus $T - \lambda I$ is {\it not} injective.  Thus $\lambda$ is an eigenvalue of $T$. \qed

%\pagebreak
%\begin{thebibliography}{99}
%
%\bibitem{Abrams1997b}
%Abrams, P.~A. and Matsuda, H.
%Prey Adaptation as a Cause of Predator-Prey Cycles.
%\emph{Evolution}
%1997, 51:1742-1750.
%
%\bibitem{Chavez2001}
%Brauer, F., Castillo-Chavez, C.
%Mathematical Models in Population Biology and Epidemiology.
%Springer,
%2011. Print.
%
%\bibitem{Boyce2012}
%Boyce, W. E., and DiPrima, R. C.
%Elementary Differential Equations and Boundary Value Problems %10\textsuperscript{th} ed.
%Wiley Global Education
%2012. Print.
%
%\bibitem{Saloniemi1993}
%Saloniemi, I.
%A Coevolutionary Predator-Prey Model with Quantitative Characters.
%\emph{American Naturalist}
%1993, 141:880-896.
%
%\bibitem{Schreiber2011}
%Schreiber, S.~J., B$\ddot{\mbox{u}}$rger,  R., and Bolnick,  D.~I.
%The Community Effects of Phenotypic and Genetic Variation within a Predator %Population.
%\emph{Ecology}
%2011,  92(8):526-543. 
%
%\end{thebibliography}

\end{document}