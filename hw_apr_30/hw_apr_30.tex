\documentclass[12pt]{article}
\textwidth=17cm \oddsidemargin=-0.9cm \evensidemargin=-0.9cm
\textheight=23.7cm \topmargin=-1.7cm

\usepackage{amssymb, amsmath, amsfonts}
\usepackage{moreverb}
\usepackage{graphicx}
\usepackage{enumerate}
\usepackage{graphics}
\usepackage{color}
\usepackage{array}
\usepackage{float}
\usepackage{hyperref}
\usepackage{textcomp}
\usepackage{alltt}
\usepackage{mathtools}
\usepackage{tikz}
\usetikzlibrary{positioning}
\usetikzlibrary{arrows}
\usepackage{pgfplots}
\usepackage{bigints}
\allowdisplaybreaks

\newcommand{\suchthat}{\, \mid \,}
\renewcommand{\theenumi}{\alph{enumi}}
\newcommand\Wider[2][3em]{%
\makebox[\linewidth][c]{%
  \begin{minipage}{\dimexpr\textwidth+#1\relax}
  \raggedright#2
  \end{minipage}%
  }%
}

\setcounter{section}{-1}

\title{\bf HW: Section 7B \#6, 7, 8, 9}
\author{\bf Sam Fleischer}
\date{\bf April 28, 2015}

\begin{document}
{\bf MATH 462 \hfill Advanced Linear Algebra \ \ \ \ \ \hfill Spring 2015} 

{\let\newpage\relax\maketitle}

\section*{Section 7B}
\subsection*{\#6}
{\it Prove that a normal operator on a complex inner product space is self-adjoint if and only if all of its eigenvalues are real.} \\

\noindent Since $T$ is normal, the Complex Spectral Theorem states that $T$ has a basis of orthonormal eigenvectors.  Let $e = (e_1, \dots, e_n)$ be one such basis.  Then $\mathcal{M}(T, e)$ is a diagonal matrix, and all entries on the diagonal are eigenvalues of $T$. \\

\noindent Now suppose $T$ is self-adjoint, i.e. $T = T^*$.  Then since $\mathcal{M}(T^*, e)$ is the conjugate transpose of $\mathcal{M}(T, e)$, then $\mathcal{M}(T, e)$ is its own conjugate transpose $\iff$ each entry on the diagonal is equal to its conjugate $\iff$ each entry on the diagonal of $\mathcal{M}(T, e)$ is real $\iff$ all of $T$'s eigenvalues are real. \hfill $\square$

\subsection*{\#7}
{\it Suppose $V$ is a complex inner product space and $T \in \mathcal{L}(V)$ is a normal operator such that $T^9 = T^8$.  Prove that $T$ is self-adjoint and $T^2 = T$.} \\

\noindent Since $T$ is normal, the Complex Spectral Theorem states that $T$ has a basis of orthonormal eigenvectors.  Let $e = (e_1, \dots, e_n)$ be one such basis.  Then $T(e_k) = \lambda_kv$.  for some $\lambda_k \in \mathbb{C}$.  Then,
\begin{align*}
		\lambda_k^9{e_k} = T^9({e_k}) &= T^8({e_k}) = \lambda_k^8{e_k} \\
		\implies \lambda_k^9{e_k} - \lambda_k^8{e_k} &= 0 \\
		\implies \lambda_k^8(\lambda_k - 1){e_k} &= 0 \\
		\implies {e_k} = 0\ \ \text{or}\ \ \lambda_k &= 0\ \ \text{or}\ \ \lambda_k = 1
\end{align*}
However, since ${e_k}$ is an eigenvector, ${e_k} \neq 0$.  Then $\lambda_k = 0$ or $\lambda_k = 1$.  Since $0$ and $1$ are real, $T$ may only have real eigenvalues.  By the previous problem, any normal operator with real eigenvalues is self-adjoint.  Thus $T$ is self-adjoint.  Also, since $0^2 = 0$ and $1^2 = 1$, then
\begin{align*}
		T^2({e_k}) = \lambda_k^2{e_k} = \lambda_k {e_k} = T({e_k})
\end{align*}
for all $k = 1, \dots, n$.  Then let $v = \displaystyle\sum\limits_{k = 1}^{n}(a_ke_k)$.  Then
\begin{align*}
		T^2(v) &= T^2\left(\sum_{k = 1}^{n}(a_ke_k)\right) = \sum_{k = 1}^{n}(a_kT^2(e_k)) = \sum_{k = 1}^{n}(a_kT(e_k)) = T\left(\sum_{k = 1}^{n}(a_ke_k)\right) = T(v)
\end{align*}  Thus $T^2 = T$. \hfill $\square$

\subsection*{\#8}
{\it Give an example of an operator $T$ on a complex vector space such that $T^9 = T^8$ but $T^2 \neq T$.} \\

\noindent Consider $T \in \mathcal{L}(\mathbb{C}^2)$ defined by $T(z_1, z_2) = (0, z_1)$.  Then $T^2(z_1, z_2) = (0, 0) \implies T^2 = \bf{0}$, where $\bf{0}$ is the zero operator.  Thus $T$ is nilpotent.  Then $T^k = {\bf0}\ \forall k \geq \text{dim}(\mathbb{C}^2) = 2$.  Thus $T^9 = T^8 = {\bf 0}$. \hfill $\square$

\subsection*{\#9}
{\it Suppose $V$ is a complex inner product space.  Prove that every normal operator on $V$ has a square root.  (An operator $S \in \mathcal{L}(V)$ is called a square root of $T \in \mathcal{L}(V)$ if $S^2 = T$.)} \\

\noindent Since $T$ is normal, the Complex Spectral Theorem states that $T$ has a basis of orthonormal eigenvectors.  Let $\omega = (\omega_1, \dots, \omega_n)$ be one such basis.  Then $T(\omega_k) = \lambda_k\omega_k$ where $\lambda_k = r_ke^{i\theta_k}$, and
\begin{align*}
		\mathcal{M}(T, \omega) = \left(\begin{array}{cccc}
				\lambda_1 & 0 & \dots & 0 \\
				0 & \lambda_2 & & \vdots \\
				\vdots & & \ddots & 0 \\
				0 & \dots & 0 & \lambda_n
		\end{array}\right)
\end{align*}
These $\lambda_k$'s need not be unique.  Since every complex number has a square root (in particular $\sqrt{\lambda_k} = (r_ke^{i\theta_k})^{\frac{1}{2}} = r_ke^{\frac{1}{2}i\theta_k}$), we can define $S \in \mathcal{L}(V)$ by where it maps each element in the basis $\omega$.
\begin{align*}
		S(\omega_k) = \sqrt{\lambda_k}\omega_k
\end{align*}
In other words,
\begin{align*}
		\mathcal{M}(S, \omega) = \left(\begin{array}{cccc}
				\sqrt{\lambda_1} & 0 & \dots & 0 \\
				0 & \sqrt{\lambda_2} & & \vdots \\
				\vdots & & \ddots & 0 \\
				0 & \dots & 0 & \sqrt{\lambda_n}
		\end{array}\right)
\end{align*}
Since $\mathcal{M}(S, \omega)$ is a diagonal matrix, $[\mathcal{M}(S, \omega)]^2$ is easily computable: $[\mathcal{M}(S, \omega)]^2 = \mathcal{M}(T, \omega)$.  Thus $S^2 = T$.  Thus every normal operator on a complex inner product space has a square root. \hfill $\square$

\end{document}
